\documentclass[letterpaper]{article}
\renewcommand{\subparagraph}{\paragraph}
\usepackage{theapa}
\usepackage{times}
\usepackage{amsmath, amsthm, amssymb}
\usepackage{pstricks}
\usepackage{pst-tree}
%\usepackage{color}
%\usepackage{makeidx}  % allows for indexgeneration
\usepackage{bm}
\usepackage{graphicx}
\usepackage{epsfig}
\usepackage{subfigure}
\usepackage{amsmath}
\usepackage{amsfonts}
\usepackage{color}
\usepackage{array}
\usepackage{colortbl}
\usepackage{framed}
\usepackage{url}
\usepackage{booktabs}
\usepackage{multirow}
%\usepackage{sgame}
\usepackage{dsfont}
\def\sgtextcolor{black}
\def\sglinecolor{black}
%\renewcommand{\gamestretch}{2}
\usepackage{multicol}
\usepackage{lscape}
\usepackage{relsize}

% ============== Mike's commands ==============
\usepackage{nicefrac}
\newcommand{\vect}[1]{\underline{\smash{#1}}}
\renewcommand{\v}[1]{\vect{#1}}
\newcommand{\reals}{\mathds{R}}
\newcommand{\sX}{\mathcal{X}}
\newcommand{\sD}{\mathcal{D}}
\newcommand{\br}{^{\text{\textnormal{ r}}}}
\newcommand{\cat}{^{\text{\textnormal{c}}}}
% ============== ==============

\newcommand{\cut}[1]{}
\renewcommand{\blue}[1]{{\textcolor{blue}{#1}}}
%\renewcommand{\blue}[1]{#1}
\usepackage{rotating}

%% Zapf Chancery font: has lowercase script letters
\DeclareFontFamily{OT1}{pzc}{}
\DeclareFontShape{OT1}{pzc}{m}{it}{<-> s * [1.200] pzcmi7t}{}
\DeclareMathAlphabet{\mathscr}{OT1}{pzc}{m}{it}

\newcommand\transpose{{\textrm{\tiny{\sf{T}}}}}
\newcommand{\note}[1]{}
\newcommand{\hlinespace}{~\vspace*{-0.15cm}~\\\hline\\\vspace*{0.15cm}}
%\newcommand{\hlinespace}{~\vspace*{0.45cm}\\\hline\\~\vspace*{-0.9cm}}
%\newcommand{\hlinespace}{~\vspace*{0.05cm}\\\hline~\vspace*{0.5cm}}

% comment the next line to turn off notes
\renewcommand{\note}[1]{~\\\frame{\begin{minipage}[c]{\textwidth}\vspace{2pt}\center{#1}\vspace{2pt}\end{minipage}}\vspace{3pt}\\}
\newcommand{\lnote}[1]{\note{#1}}
\newcommand{\emcite}[1]{\citet{#1}}
\newcommand{\yrcite}[1]{\citeyear{#1}}
\newcommand{\aunpcite}[1]{\citeR{#1}}

\newcommand{\heavyrule}{\specialrule{\heavyrulewidth}{.4em}{.4em}}
\newcommand{\lightrule}{\specialrule{.03em}{.4em}{.4em}}

%%%%%%%%%%%%%%%%%%%%%%%%%%%%%%%%%%%%%%%%%%%%%%

%% keep figures from going onto a page by themselves
\renewcommand{\topfraction}{0.9}
\renewcommand{\textfraction}{0.07}
\renewcommand{\floatpagefraction}{0.9}
\renewcommand{\dbltopfraction}{0.9}      % for double-column styles
\renewcommand{\dblfloatpagefraction}{0.7}   % for double-column styles



\usepackage{amsmath, amsthm, amssymb}
\newtheorem{thm}{Theorem}%[section]
\newtheorem{lem}[thm]{Lemma}
\newtheorem{prop}[thm]{Proposition}
\newtheorem{cor}[thm]{Corollary}
\newtheorem{obs}[thm]{Observation}

%\theoremstyle{definition}
\newtheorem{define}[thm]{Definition}
\hyphenation{ge-ne-ral-ize}



%SAT solvers
\newcommand{\spear}{\algofont{SPEAR}}
\newcommand{\saps}{\algofont{SAPS}}
\newcommand{\satenstein}{\algofont{SATenstein}}
\newcommand{\tnm}{\algofont{tnm}}
\newcommand{\minisat}{\algofont{Minisat~2.0}}
\newcommand{\lkh}{\algofont{LK-H}}
\newcommand{\concorde}{\algofont{Concorde}}
\newcommand{\cryptominisat}{\algofont{CryptoMinisat}}
\newcommand{\satelite}{\algofont{SATElite}}

%MIP solvers
\newcommand{\lpsolve}{\algofont{lp\_solve}}
\newcommand{\scip}{\algofont{SCIP}}
\newcommand{\gurobi}{\algofont{Gu\-ro\-bi}}
\newcommand{\cplex}{\algofont{CPLEX}}
\newcommand{\ibmcplex}{\algofont{IBM ILOG CPLEX}}

\newcommand{\SATzilla}{\texttt{SATzilla}}
\newcommand{\SATzillaHHM}{\texttt{SATzilla$\_$HHM}}
\newcommand{\SATzillaNew}{\texttt{SATzilla07}}
\newcommand{\SATzillap}{\texttt{SATzilla07$^+$}}
\newcommand{\SATzillas}{\texttt{SATzilla07$^*$}}

\newcommand{\random}{\texttt{RANDOM}}
\newcommand{\industrial}{\texttt{INDUSTRIAL}}
\newcommand{\crafted}{\texttt{CRAFTED}}
\newcommand{\handmade}{\texttt{HANDMADE}}
\newcommand{\all}{\texttt{ALL}}

\newcommand{\Var}{\ensuremath\text{Var}}
\newcommand{\indicator}{\ensuremath\mathds{I}}

\newcommand{\fhspace}{\vspace*{0.2cm}}
\newcommand{\newsec}{\hspace{0cm}}

% replaces tabular; takes same arguments. use \midrule for a rule, no vertical rules, and eg \cmidrule(l){2-3} as needed with \multicolumn
\newenvironment{ktabular}[1]{\sffamily\small\begin{center}\begin{tabular}[c]{#1}\toprule}{\bottomrule \end{tabular}\end{center}\normalsize\rmfamily\vspace{-5pt}}
\newcommand{\tbold}[1]{\textbf{#1}}
\newcommand{\interrowspace}{.6em}


\begin{document}

\title{A Kernel for Hierarchical Parameter Spaces}

\author{Frank Hutter and Michael A. Osborne\\
{\tt fh@informatik.uni-freiburg.de} and {\tt mosb@robots.ox.ac.uk}
}

\maketitle
\begin{abstract}
\noindent{}We define a family of kernels for mixed continuous/discrete hierarchical parameter spaces and show that they are positive definite.
\end{abstract}

%%%%%%%%%%%%%%%%%%%%%%%%%%%%%%%%%%%%%%%%%%%%%%%%%%%%%%%%%%%%%%%%%%%%%%%%%%%%%%%%%%%%%%%%%%%%%%%%%%%%%%%%%%%%%%%%%%%%%%%%%%%%%
\section{Introduction}
%%%%%%%%%%%%%%%%%%%%%%%%%%%%%%%%%%%%%%%%%%%%%%%%%%%%%%%%%%%%%%%%%%%%%%%%%%%%%%%%%%%%%%%%%%%%%%%%%%%%%%%%%%%%%%%%%%%%%%%%%%%%%

We aim to do inference about some function $g$ with domain (input space) $\sX$. $\sX = \prod_{i=1}^D \sX_i$ is a $D$-dimensional input space, where each individual dimension is either bounded real or categorical, that is, $\sX_i$ is either $[l_i, u_i] \subset \reals$ (with lower and upper bounds $l_i$ and $u_i$, respectively) or $\{v_{i,1}, \dots, v_{i,m_i}\}$. 

Associated with $\sX$, there is a DAG structure $\sD$, whose vertices are the dimensions $\{1,\,\ldots,\,D\}$. $\sX$ will be restricted by $\sD$: if vertex $i$ has children under $\sD$, $\sX_i$ must be categorical. $\sD$ is also used to specify when each input is \emph{active} (that is, relevant to inference about $g$). In particular, we assume each input dimension is only active under some instantiations of its ancestor dimensions in $\sD$. More precisely, we define $D$ functions $\delta_i\colon \sX\to \mathcal{B}$, for $i \in \{1,\,\ldots,\,D\}$, and where $\mathcal{B} = \{\text{true}, \text{false}\}$. We take 
\begin{equation}
 \delta_i(\v{x}) = \delta_i\bigl(\v{x}(\text{anc}_i)\bigr),
\end{equation}
where $\text{anc}_i$ are the ancestor vertices of $i$ in $\sD$, such that $\delta_i(\v{x})$ is true only for appropriate values of those entries of $\v{x}$ corresponding to ancestors of $i$ in $\sD$. We say $i$ is active for $\v{x}$ iff $\delta_i(\v{x})$.

%if all its parent dimensions $P_i$ are active themselves and each parent $p\in P_i$ takes one of the values in the finite set $V_{i,p}$. 
Our aim is to specify a kernel for $\sX$, \emph{i.e.}, a positive semi-definite function  $k\colon \sX \times \sX \to \reals$. We will first specify an individual kernel for each input dimension, \emph{i.e.}, a positive semi-definite function $k_i\colon \sX \times \sX \to \reals$. $k$ can then be taken as either a sum,
\begin{equation}
 k(\v{x}, \v{x}') = \sum_{i=1}^D k_i(\v{x},\v{x}'),
\end{equation}
product,
\begin{equation}
 k(\v{x}, \v{x}') = \prod_{i=1}^D k_i(\v{x},\v{x}'),
\end{equation}
or any other permitted combination, of these individual kernels. Note that each individual kernel $k_i$ will depend on an input vector $\v{x}$ only through dependence on $x_i$ and $\delta_i(\v{x})$,
\begin{equation}
  k_i(\v{x},\v{x}') = \tilde{k}_i\bigl(x_i,\delta_i(\v{x}),x_i', \delta_i(\v{x}') \bigr).
\end{equation}
That is, $x_j$ for $j\neq i$ will influence $k_i(\v{x},\v{x}')$ only if $j \in \text{anc}_i$, and only by affecting whether $i$ is active.

Below we will construct pseudometrics $d{_i}\colon \sX \times \sX \to \reals^+$: that is, $d_i$ satisfies the requirements of a metric aside from the identity of indiscernibles. As for $k_i$, these pseudometrics will depend on an input vector $\v{x}$ only through dependence on both $x_i$ and $\delta_i(\v{x})$. $d{_i}(\v{x}, \v{x}')$ will be designed to provide an intuitive measure of how different $g(\v{x})$ is from $g(\v{x}')$. 
For each $i$, we will then construct a (pseudo-)isometry $f_i$ from
$\sX$ 
to a Euclidean space ($\reals^2$ for bounded real parameters, and $\reals^m$ for categorical-valued parameters with $m$ choices). That is, denoting the Euclidean metric on the appropriate space as $d{_E}$, $f_i$ will be such that
\begin{equation}
\label{eqn:d_i}
 d{_i}(\v{x},\v{x}')
=
d_{\text{E}}(f{_i}\bigl(\v{x}), f{_i}(\v{x}')\bigr)
\end{equation}
for all $\v{x}, \v{x}' \in \sX$. We can then use our transformed inputs, $f_i(\v{x})$, within any standard Euclidean kernel $\kappa$. We'll make this explicit in Proposition \ref{prop:psd_if_isometry}. 

\begin{define}
\label{def:psd_fun_euclid}
A function $\kappa\colon \reals^+ \to \reals$ is \emph{a positive semi-definite covariance function over Euclidean space} if $K \in \reals^{N\times N}$, defined by 
\begin{equation}
\nonumber K_{m, n} = \kappa\bigl(d_{\text{E}}(\v{y}_m, \v{y}_n)\bigr),\quad \text{for }
\v{y}_m, \v{y}_n \in \reals^P,\quad m, n = 1, \ldots, N, 
\end{equation}
is positive semi-definite for any $\v{y}_1, \dots, \v{y}_N \in \reals^P$. 
\end{define}

A popular example of such a $\kappa$ is the exponentiated quadratic, for which $\kappa(\delta) = \sigma^2 \exp(-\frac{1}{2} \frac{\delta^2}{\lambda^2})$; another popular choice is the rational quadratic, for which $\kappa(\delta) = \sigma^2 (1+\frac{1}{2\alpha} \frac{\delta^2}{\lambda^2})^{-\alpha}$.


\begin{prop}
Let $\kappa$ be a positive semi-definite covariance function over Euclidean space and let $d_i$ satisfy Equation \ref{eqn:d_i}. Then, 
$k_i\colon \sX \times \sX\to \reals^+$, defined by 
%\[k_i(\v{x},\v{x}') = \kappa( d_{\text{E}}(f_i(\v{x}), f_i(\v{x}')) )\]
\[k_i(\v{x},\v{x}') = \kappa\bigl( d_i(\v{x}, \v{x}') \bigr)\]
is a positive semi-definite covariance function over input space $\sX$. 
\label{prop:psd_if_isometry}
\begin{proof}
We need to show that for any $\v{x}_1, \dots, \v{x}_N \in \sX$, $K \in \reals^{N\times N}$ defined by
\begin{align*}
 K_{m, n} & = \kappa\bigl(d{_i}(\v{x}_m,\v{x}_n)\bigr)
,\quad \text{for }
\v{x}_m, \v{x}_n \in \sX,\quad m, n = 1, \ldots, N, 
\\
\intertext{is positive semi-definite. Now, by the definition of $d_i$,}
K_{m, n} & = \kappa\Bigl(d_{\text{E}}(f{_i}\bigl(\v{x}_m), f{_i}(\v{x}_n)\bigr)\Bigr) 
= \kappa\bigl(d_{\text{E}}(\v{y}_m, \v{y}_n)\bigr)
\end{align*}
where $\v{y}_m = f{_i}\bigl(\v{x}_m)$ and $\v{y}_n = f{_i}\bigl(\v{x}_n)$ are elements of $\reals^P$.
Then, by assumption that $\kappa$ is a positive semi-definite covariance function over Euclidean space, $K$ is positive semi-definite. 
\end{proof}
\end{prop}

We'll now define pseudometrics $d_i$ and associated isometries $f_i$ for both the bounded real and categorical cases. 


%%%%%%%%%%%%%%%%%%%%%%%%%%%%%%%%%%%%%%%%%%%%%%%%%%%%%%%%%%%%%%%%%%%%%%%%%%%%%%%%%%%%%%%%%%%%%%%%%%%%%%%%%%%%%%%%%%%%%%%%%%%%%
\section{Bounded Real Dimensions}
%%%%%%%%%%%%%%%%%%%%%%%%%%%%%%%%%%%%%%%%%%%%%%%%%%%%%%%%%%%%%%%%%%%%%%%%%%%%%%%%%%%%%%%%%%%%%%%%%%%%%%%%%%%%%%%%%%%%%%%%%%%%%

Let's first focus on a bounded real input dimension $i$, i.e., $\sX_i=[l_i, u_i]$.
To emphasize that we're in this real case, we explicitly denote the pseudometric as $d\br_i$ and the (pseudo-)isometry from $(\sX, d_i)$ to $\reals^2,d_\text{E}$ 
as $f\br_i$. For the definitions, recall that $\delta_i(\v{x})$ is true iff dimension $i$ is active given the instantiation of $i$'s ancestors in $\v{x}$.

\begin{eqnarray}
\nonumber{}d\br_i(\v{x}, \v{x}') & = & \left\{\begin{array}{ll}
\nonumber{} 0 & \textrm{ if } \delta_i(\v{x}) = \delta_i(\v{x}') = \textrm{false}\\
\nonumber{} \omega_i & \textrm{ if } \delta_i(\v{x}) \neq \delta_i(\v{x}')\\
\nonumber{} \omega_i \sqrt{2} \sqrt{1 - \cos(\pi\rho_i \frac{x_i-x_i'}{u_i-l_i})} & \textrm{ if } \delta_i(\v{x}) = \delta_i(\v{x}') = \textrm{true}. \end{array}\right.
\end{eqnarray}

\begin{eqnarray}
\nonumber{}f_i\br(\v{x}) & = & \left\{\begin{array}{ll}
[0,0]^\transpose & \textrm{ if } \delta_i(\v{x}) = \textrm{ false }\\
\nonumber{} \omega_i [\sin{\pi\rho_i\frac{x_i}{u_i-l_i}}, \cos{\pi\rho_i\frac{x_i}{u_i-l_i}}]^\transpose & \textrm{ otherwise.}\end{array}\right..
\end{eqnarray}

Although our formal arguments do not rely on this, Proposition \ref{prop:dbr_pseudometric} in the appendix shows that $d\br_i$ is a pseudometric. 
This pseudometric is defined by two parameters: $\omega_i \in [0,1]$ and $\rho_i \in [0,1]$. We firstly define 
\begin{equation}
\omega_i = \prod_{j \in \text{anc}_i \cup \{i\}} \gamma_j, 
\end{equation}
where $\gamma_j \in [0,1]$. This encodes the intuitive notion that differences on lower levels of the hierarchy count less than differences in their ancestors.
\note{FH: added $\gamma_i$ in the product for $w_i$, so we can parameterize the weight of root nodes.}

Also note that, as desired, if $i$ is inactive for both $\v{x}$ and $\v{x}'$, $d\br_i$ specifies that $g(\v{x})$ and $g(\v{x}')$ should not differ owing to differences between $x_i$ and $x_i'$. Secondly, if $i$ is active for both $\v{x}$ and $\v{x}'$, the difference between $g(\v{x})$ and $g(\v{x}')$ due to $x_i$ and $x_i'$ increases monotonically with increasing $\left|x_i-x_i'\right|$. Parameter $\rho_i$ controls whether differing in the activity of $i$ contributes more or less to the distance than differing in $x_i$ should $i$ be active. If $\rho = \nicefrac{1}{3}$, and if $i$ is inactive for exactly one of $\v{x}$ and $\v{x}'$, $g(\v{x})$ and $g(\v{x}')$ are as different as is possible due to dimension $i$; that is, $g(\v{x})$ and $g(\v{x}')$ are exactly as different in that case as if $x_i=l_i$ and $x_i'=u_i$. For $\rho>\nicefrac{1}{3}$, $i$ being active for both $\v{x}$ and $\v{x}'$ means that $g(\v{x})$ and $g(\v{x}')$ could potentially be more different than if
$i$ was active in only one of them. For $\rho<\nicefrac{1}{3}$, the converse is true.\footnote{Note that $\v{x}$ and $\v{x}'$ must differ in at least one ancestor dimension of $i$ in order for $\delta_i(\v{x}) \neq \delta_i(\v{x}')$ to hold, such that in the final kernel combining kernels $k_i$ due to each dimension $i$, differences in the activity of dimension $i$ are penalized both in kernel $k_i$ and in the distance for the kernel of the ancestor dimension causing the difference in $i$'s activity.
\note{FH: note: if we wanted to, we could probably get a condition for the joint overall kernel that the distance will always be larger if configurations differ on a higher level in the dimensionality DAG, by multiplying $\omega_i$ by the weights $\omega_j \in [0,1]$ of all of $i$'s ancestors $j$ (because at least one ancestor has to differ). We'd further have to divide $\omega_i$ by the maximal number of descendants a dimension has, in order to ensure that a difference at a higher level counts more than all the differences at the descendant-dimensions combined. But of course for this we wouldn't be able to do the proof of PSD-ness for each dimension by itself, so things would get a lot more hairy, and at least at this point there is no need for that.\\MO: actually, we should absolutely be able to do that, without any problem. I don't think it'll get hairy. It's just a matter of replacing $\omega_i$ with a slightly different constant, dependent on $i$, but independent of $\v{x}$. Let's do it!\\FH: I agree with what you did, and it's nice! What I meant would be hairy is to not penalize differences in activity of $i$ at all on the level of $i$, but to do it all on the ancestor level. For that each individual dimension would not define a kernel anymore, so the current way of proving PSD-ness wouldn't work anymore, and in general, thinking more about it now, the end result might not even be PSD. But enough of me going on about this, what we have now is nice :-) Please feel free to delete the note ...}}

We now show that $d\br_i$ and $f\br_i$ can be plugged into a positive semi-definite kernel over Euclidean space to define a valid kernel over space $\sX$.

\begin{prop}
Let $\kappa$ be a positive semi-definite covariance function over Euclidean space.
Then, $k_i\colon \sX \times \sX\to \reals^+$, defined by 
%\[k_i(\v{x},\v{x}') = \kappa( d_{\text{E}}(f_i(\v{x}), f_i(\v{x}')) )\]
\[k_i(\v{x},\v{x}') = \kappa\bigl( d\br_i(\v{x}, \v{x}') \bigr)\]
is a positive semi-definite covariance function over input space $\sX$. 
\label{prop:cont_psd}
\begin{proof}
Due to Proposition \ref{prop:psd_if_isometry}, we only need to show that, for any two inputs $\v{x},\v{x}' \in \sX$, the isometry condition $d_{\text{E}}\bigl(f_i\br(\v{x}),f_i\br(\v{x}')\bigr) = d\br_i(\v{x},\v{x}')$ holds.

We use the abbreviation $\alpha = \pi\rho_i\frac{x_i}{u_i-l_i}$ and $\alpha' = \pi\rho_i\frac{x'_i}{u_i-l_i}$ and consider the following three possible cases of dimension $i$ being active or inactive in $\v{x}$ and $\v{x}'$.

~\\\noindent{}{Case 1}: $\delta_i(\v{x}) = \delta_i(\v{x}') = \textrm{false}$.
In this case, we trivially have 
\[d_{\text{E}}(f_i\br(\v{x}),f_i\br(\v{x}')) = d_{\text{E}}([0,0]^\transpose, [0,0]^\transpose) = 0 = d\br_i(\v{x},\v{x}').\]

~\\\noindent{}{Case 2}: $\delta_i(\v{x}) \neq \delta_i(\v{x}')$. In this case, we have
\[d_{\text{E}}(f_i\br(\v{x}),f_i\br(\v{x}')) = d_{\text{E}}([\sin{\alpha}, \cos{\alpha}]^\transpose, [0,0]^\transpose) = \sqrt{\omega_i^2 (\sin^2{\alpha} + \cos^2{\alpha})} = \omega_i = d\br_i(\v{x},\v{x}'),\]
and symmetrically for $d_{\text{E}}([0,0]^\transpose, [\sin{\alpha}, \cos{\alpha}]^\transpose)$.

~\\\noindent{}{Case 3}: $\delta_i(\v{x}) = \delta_i(\v{x}') = \textrm{true}$. We have:
\begin{eqnarray}
\nonumber{}d_{\text{E}}(f_i\br(\v{x}),f_i\br(\v{x}')) & = & d_{\text{E}}(\omega_i [\sin{\alpha}, \cos{\alpha}]^\transpose, \omega_i [\sin{\alpha'}, \cos{\alpha'}]^\transpose)\\ 
\nonumber{}& = & \omega_i \sqrt{(\sin{\alpha}-\sin{\alpha'})^2+ (\cos{\alpha}-\cos{\alpha'})^2}\\
\nonumber{}& = & \omega_i \sqrt{\sin^2{\alpha} -2 \sin{\alpha}\sin{\alpha'} + \sin^2{\alpha'}  + \cos^2{\alpha} -2 \cos{\alpha}\cos{\alpha'} + \cos^2{\alpha'} }\\
\nonumber{}& = & \omega_i \sqrt{(\sin^2{\alpha}+\cos^2{\alpha})   +  (\sin^2{\alpha'}+\cos^2{\alpha'})   -2 (\sin{\alpha}\sin{\alpha'} + \cos{\alpha}\cos{\alpha'})}\\
\label{eqn:simplified}& = & \omega_i \sqrt{ 1+1-2 \cos(\alpha-\alpha')}\\
\nonumber& = & \omega_i \sqrt{2} \sqrt{1 - \cos(\pi\rho_i \frac{x_i-x_i'}{u_i-l_i})} = d\br_i(\v{x}, \v{x}'),
\end{eqnarray}
where (\ref{eqn:simplified}) follows from the previous line by using the identity 
\[\cos{(a-b)} = \cos{a}\cos{b} + \sin{a}\sin{b}.\]
\end{proof}
\end{prop}


%%%%%%%%%%%%%%%%%%%%%%%%%%%%%%%%%%%%%%%%%%%%%%%%%%%%%%%%%%%%%%%%%%%%%%%%%%%%%%%%%%%%%%%%%%%%%%%%%%%%%%%%%%%%%%%%%%%%%%%%%%%%%
\section{Categorical Dimensions}
%%%%%%%%%%%%%%%%%%%%%%%%%%%%%%%%%%%%%%%%%%%%%%%%%%%%%%%%%%%%%%%%%%%%%%%%%%%%%%%%%%%%%%%%%%%%%%%%%%%%%%%%%%%%%%%%%%%%%%%%%%%%%

Now let's define $f\cat_i$ and $d\cat_i$ for the case that the input $\sX_i=\{v_{i,1}, \dots, v_{i,m_i}\}$ is categorical with $m_i$ possible values. 
Proceeding as above, we define a pseudometric $d\cat_i$ on $\sX$ and an isometry from $(\sX, d\cat_i)$ to $(\reals^{m_i},d_{\text{E}}^{m_i})$, and show that we can use combine these
with a kernel over Euclidean space to construct a valid kernel over space $\sX$. 

\begin{eqnarray}
\nonumber{}d\cat_i(\v{x}, \v{x}') & = & \left\{
\begin{array}{ll}
\nonumber{} 0 & \textrm{ if } \delta_i(\v{x}) = \delta_i(\v{x}') = \textrm{false}\\
\nonumber{} \omega_i & \textrm{ if } \delta_i(\v{x}) \neq \delta_i(\v{x}')\\
\nonumber{} \omega_i \sqrt{2} \indicator_{x_i \neq x_i'} 
& \textrm{ if } \delta_i(\v{x}) = \delta_i(\v{x}') = \textrm{true}.
\end{array}
\right.
\end{eqnarray}

\begin{eqnarray}
\nonumber{}f\cat_i(\v{x}) & = & \left\{\begin{array}{ll}
\v{0} \in \reals^{m_i} & \textrm{ if } \delta_i(\v{x}) = \textrm{ false }\\
\nonumber{} \omega_i\,\frac{\v{e_j}+(1-\rho)\sum_{l\neq j} \v{e_l}}
{\|\v{e_j}+(1-\rho)\sum_{l\neq j} \v{e_l}\|}
 & \delta_i(\v{x}) = \textrm{ true and } x_i = v_{i,j},
\end{array}\right.
\end{eqnarray}
\noindent{}where $\v{e_j} \in \reals^{m_i}$ is zero in all dimensions except $j$, where it it $1$.


\begin{prop}
Let $\kappa$ be a positive semi-definite covariance function over Euclidean space.
Then, $k_i\colon \sX \times \sX\to \reals^+$, defined by 
%\[k_i(\v{x},\v{x}') = \kappa( d_{\text{E}}(f_i(\v{x}), f_i(\v{x}')) )\]
\[k_i(\v{x},\v{x}') = \kappa\bigl( d\cat_i(\v{x}, \v{x}') \bigr)\]
is a positive semi-definite covariance function over input space $\sX$. 
\label{prop:cat_psd}
\begin{proof}
We proceed as in the proof of Proposition \ref{prop:cont_psd} to show that, for any two inputs $\v{x},\v{x}' \in \sX$, the isometry condition $d_{\text{E}}^{m_i}(f\cat_i(\v{x}),f\cat_i(\v{x}')) = d\cat_i(\v{x},\v{x}')$ holds.

~\\\noindent{}{Case 1}: $\delta_i(\v{x}) = \delta_i(\v{x}') = \textrm{false}$.
In this case, we trivially have 
\[d_{\text{E}}^{m_i}(f_i\br(\v{x}),f_i\br(\v{x}')) = d_{\text{E}}^{m_i}(\v{0}, \v{0}) = 0 = d_i\br(\v{x},\v{x}').\]

~\\\noindent{}{Case 2}: $\delta_i(\v{x}) \neq \delta_i(\v{x}')$. In this case, we have
\[d_{\text{E}}^{m_i}(f\cat_i(\v{x}),f\cat_i(\v{x}')) = d_{\text{E}}^{m_i}(\omega_i\,\v{e_j}, \v{0}) = \omega_i = d_i(\v{x},\v{x}'),\]
and symmetrically for $d_{\text{E}}(\v{0}, \omega_i\,\v{e_j})$.

~\\\noindent{}{Case 3}: $\delta_i(\v{x}) = \delta_i(\v{x}') = \textrm{true}$. 
If $x_i=x_i'=v_{i,j}$, we have 
\begin{eqnarray}
\nonumber{}d_{\text{E}}^{m_i}(f\cat_i(\v{x}),f\cat_i(\v{x}')) & = & d_{\text{E}}^{m_i}(\omega_i\,\v{e_j}, \omega_i\,\v{e_j}) = 0 = d\cat_i(\v{x}, \v{x}')
\end{eqnarray}

\noindent{}If $x_i=v_{i,j} \neq v_{i,j'} = x_i'=$, we have 
\begin{eqnarray} 
\nonumber{}d_{\text{E}}(f\cat_i(\v{x}),f\cat_i(\v{x}')) & = & d_{\text{E}}^{m_i}(\omega_i\,\v{e_j}, \omega_i \v{e_{j'}}) = \omega_i \sqrt{2} = d\cat_i(\v{x}, \v{x}')
\end{eqnarray}
\end{proof}
\end{prop}


\appendix

\section{Proof of pseudometric properties}

\begin{prop}
  $d\br_i$ is a pseudometric on $\sX$. \label{prop:dbr_pseudometric}
\begin{proof}
The non-negativity and symmetry of $d\br_i$ are trivially proven. To prove the triangle inequality, consider $\v{x}, \v{x}', \v{x}'' \in \sX$. 

~\\\noindent{}{Case 1}: $\delta_i(\v{x}) = \delta_i(\v{x}') = \textrm{false}$, such that $d\br_i(\v{x},\v{x}') = 0$. Here, from non-negativity, clearly $d\br_i(\v{x},\v{x}') = 0 \leq d\br_i(\v{x},\v{x}'') + d\br_i(\v{x}',\v{x}'')$.

~\\\noindent{}{Case 2}: $\delta_i(\v{x}) \neq \delta_i(\v{x}')$, such that such that  $d\br_i(\v{x},\v{x}') = \omega_i$.  Without loss of generality, assume $\delta_i(\v{x}) = \text{true}$, $\delta_i(\v{x}') = \text{false}$ and $\delta_i(\v{x}'') = \text{true}$. 
\begin{align}
d\br_i(\v{x},\v{x}'') + d\br_i(\v{x}',\v{x}'') = d\br_i(\v{x},\v{x}'')  + \omega_i
\end{align}
Hence $d\br_i(\v{x},\v{x}'') + d\br_i(\v{x}',\v{x}'') \geq \omega_i = d\br_i(\v{x},\v{x}')$ by non-negativity.

~\\\noindent{}{Case 3}: $\delta_i(\v{x}) = \delta_i(\v{x}')=\textrm{true}$, such that  $d\br_i(\v{x},\v{x}') = \omega_i \sqrt{2} \sqrt{1 - \cos(\pi\rho_i \frac{x_i-x_i'}{u_i-l_i})}$.  If  $\delta_i(\v{x}'') = \text{false}$,
\begin{align}
d\br_i(\v{x},\v{x}'') + d\br_i(\v{x}',\v{x}'') = 2 \omega_i \geq \omega_i \sqrt{2} \sqrt{1 - \cos(\pi\rho_i \frac{x_i-x_i'}{u_i-l_i})} = d\br_i(\v{x},\v{x}').
\end{align} 
If  $\delta_i(\v{x}'') = \text{true}$, consider the worst possible case in which, without loss of generality, $x_i=l_i$ and $x'_i=u_i$, such that $d\br_i(\v{x},\v{x}')=2 \omega_i^2$.  We define the abbreviation $\beta'' = \frac{x''_i-l_i}{u_i-l_i}$, giving
\begin{align}
\bigl(d\br_i(\v{x},\v{x}'') + d\br_i(\v{x}',\v{x}'')\bigr)^2
& = 2\omega_i^2 \Bigl(\sqrt{1 - \cos (\pi\rho_i \beta'')} + \sqrt{1 - \cos \bigl(\pi\rho_i (1-\beta'')\bigr)}\Bigr)^2\nonumber\\
&=2\omega_i^2\biggl(2 - \cos (\pi\rho_i \beta'') - \cos \bigl(\pi\rho_i (1-\beta'')\bigr)
\nonumber\\
&\qquad\qquad+2\sqrt{\Bigl(1 - \cos (\pi\rho_i \beta'')\Bigr)\Bigl(1 - \cos \bigl(\pi\rho_i (1-\beta'')\bigr)\Bigr)}\biggr)\nonumber\\
&=2\omega_i^2\biggl(2 +2\sqrt{1 + \cos (\pi\rho_i \beta'')\cos \bigl(\pi\rho_i (1-\beta'')\bigr)}\biggr)\nonumber\\
&=4 \omega_i^2 \bigl(1 + \left|\sin \pi\rho_i \beta'' \right|\bigr)\nonumber\\
&\geq 4 \omega_i^2 = d\br_i(\v{x},\v{x}')^2.
\end{align}
Hence, from non-negativity, we have $d\br_i(\v{x},\v{x}'') + d\br_i(\v{x}',\v{x}'')\geq d\br_i(\v{x},\v{x}')$.
\end{proof}
\end{prop}

\begin{prop}
 $d\cat_i$ is a pseudometric on $\sX$.\label{prop:dbr_pseudometric_cat}
 \begin{proof}
 A trivial modification of the proof to Proposition \ref{prop:dbr_pseudometric}.
 \end{proof}
\end{prop}


%\bibliographystyle{theapa}
%\renewcommand{\baselinestretch}{0.97}
%\footnotesize{\bibliography{abbrev,frankbib}}


\end{document}

