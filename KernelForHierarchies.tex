\documentclass[letterpaper]{article}
\renewcommand{\subparagraph}{\paragraph}
\usepackage{theapa}
\usepackage{times}
\usepackage{amsmath, amsthm, amssymb}
\usepackage{pstricks}
\usepackage{pst-tree}
%\usepackage{color}
%\usepackage{makeidx}  % allows for indexgeneration
\usepackage{bm}
\usepackage{graphicx}
\usepackage{epsfig}
\usepackage{subfigure}
\usepackage{amsmath}
\usepackage{amsfonts}
\usepackage{color}
\usepackage{array}
\usepackage{colortbl}
\usepackage{framed}
\usepackage{url}
\usepackage{booktabs}
\usepackage{multirow}
%\usepackage{sgame}
\usepackage{dsfont}
\def\sgtextcolor{black}
\def\sglinecolor{black}
%\renewcommand{\gamestretch}{2}
\usepackage{multicol}
\usepackage{lscape}
\usepackage{relsize}

% ============== Mike's commands ==============
\newcommand{\vect}[1]{\underline{\smash{#1}}}
\renewcommand{\v}[1]{\vect{#1}}
\newcommand{\reals}{\mathds{R}}
\newcommand{\sX}{\mathcal{X}}
\newcommand{\sD}{\mathcal{D}}
\newcommand{\br}{^{(\text{br})}}
\newcommand{\cat}{^{(\text{c})}}
% ============== ==============

\newcommand{\cut}[1]{}
\renewcommand{\blue}[1]{{\textcolor{blue}{#1}}}
%\renewcommand{\blue}[1]{#1}
\usepackage{rotating}

%% Zapf Chancery font: has lowercase script letters
\DeclareFontFamily{OT1}{pzc}{}
\DeclareFontShape{OT1}{pzc}{m}{it}{<-> s * [1.200] pzcmi7t}{}
\DeclareMathAlphabet{\mathscr}{OT1}{pzc}{m}{it}

\newcommand\transpose{{\textrm{\tiny{\sf{T}}}}}
\newcommand{\note}[1]{}
\newcommand{\hlinespace}{~\vspace*{-0.15cm}~\\\hline\\\vspace*{0.15cm}}
%\newcommand{\hlinespace}{~\vspace*{0.45cm}\\\hline\\~\vspace*{-0.9cm}}
%\newcommand{\hlinespace}{~\vspace*{0.05cm}\\\hline~\vspace*{0.5cm}}

% comment the next line to turn off notes
\renewcommand{\note}[1]{~\\\frame{\begin{minipage}[c]{\textwidth}\vspace{2pt}\center{#1}\vspace{2pt}\end{minipage}}\vspace{3pt}\\}
\newcommand{\lnote}[1]{\note{#1}}
\newcommand{\emcite}[1]{\citet{#1}}
\newcommand{\yrcite}[1]{\citeyear{#1}}
\newcommand{\aunpcite}[1]{\citeR{#1}}

\newcommand{\heavyrule}{\specialrule{\heavyrulewidth}{.4em}{.4em}}
\newcommand{\lightrule}{\specialrule{.03em}{.4em}{.4em}}

%%%%%%%%%%%%%%%%%%%%%%%%%%%%%%%%%%%%%%%%%%%%%%

%% keep figures from going onto a page by themselves
\renewcommand{\topfraction}{0.9}
\renewcommand{\textfraction}{0.07}
\renewcommand{\floatpagefraction}{0.9}
\renewcommand{\dbltopfraction}{0.9}      % for double-column styles
\renewcommand{\dblfloatpagefraction}{0.7}   % for double-column styles



\usepackage{amsmath, amsthm, amssymb}
\newtheorem{thm}{Theorem}%[section]
\newtheorem{lem}[thm]{Lemma}
\newtheorem{prop}[thm]{Proposition}
\newtheorem{cor}[thm]{Corollary}
\newtheorem{obs}[thm]{Observation}

%\theoremstyle{definition}
\newtheorem{define}[thm]{Definition}
\hyphenation{ge-ne-ral-ize}



%SAT solvers
\newcommand{\spear}{\algofont{SPEAR}}
\newcommand{\saps}{\algofont{SAPS}}
\newcommand{\satenstein}{\algofont{SATenstein}}
\newcommand{\tnm}{\algofont{tnm}}
\newcommand{\minisat}{\algofont{Minisat~2.0}}
\newcommand{\lkh}{\algofont{LK-H}}
\newcommand{\concorde}{\algofont{Concorde}}
\newcommand{\cryptominisat}{\algofont{CryptoMinisat}}
\newcommand{\satelite}{\algofont{SATElite}}

%MIP solvers
\newcommand{\lpsolve}{\algofont{lp\_solve}}
\newcommand{\scip}{\algofont{SCIP}}
\newcommand{\gurobi}{\algofont{Gu\-ro\-bi}}
\newcommand{\cplex}{\algofont{CPLEX}}
\newcommand{\ibmcplex}{\algofont{IBM ILOG CPLEX}}

\newcommand{\SATzilla}{\texttt{SATzilla}}
\newcommand{\SATzillaHHM}{\texttt{SATzilla$\_$HHM}}
\newcommand{\SATzillaNew}{\texttt{SATzilla07}}
\newcommand{\SATzillap}{\texttt{SATzilla07$^+$}}
\newcommand{\SATzillas}{\texttt{SATzilla07$^*$}}

\newcommand{\random}{\texttt{RANDOM}}
\newcommand{\industrial}{\texttt{INDUSTRIAL}}
\newcommand{\crafted}{\texttt{CRAFTED}}
\newcommand{\handmade}{\texttt{HANDMADE}}
\newcommand{\all}{\texttt{ALL}}

\newcommand{\Var}{\ensuremath\text{Var}}
\newcommand{\indicator}{\ensuremath\mathds{I}}

\newcommand{\fhspace}{\vspace*{0.2cm}}
\newcommand{\newsec}{\hspace{0cm}}

% replaces tabular; takes same arguments. use \midrule for a rule, no vertical rules, and eg \cmidrule(l){2-3} as needed with \multicolumn
\newenvironment{ktabular}[1]{\sffamily\small\begin{center}\begin{tabular}[c]{#1}\toprule}{\bottomrule \end{tabular}\end{center}\normalsize\rmfamily\vspace{-5pt}}
\newcommand{\tbold}[1]{\textbf{#1}}
\newcommand{\interrowspace}{.6em}


\begin{document}

\title{A Kernel for Hierarchical Parameter Spaces}

\author{Frank Hutter and Michael A. Osborne\\
{\tt hutter@cs.ubc.ca} and {\tt mosb@robots.ox.ac.uk}
%     Department of Computer Science\\
%     University of British Columbia \\
%     201-2366 Main Mall, BC V6T 1Z4, CANADA\\
%     {\texttt{hutter@cs.ubc.ca}}
}

\maketitle
\begin{abstract}
\noindent{}We define kernels for mixed continuous/discrete spaces and conditional spaces and show that they are positive definite.
\end{abstract}

We aim to do inference about some function $g$ with domain (input space) $\sX$. $\sX = \prod_{i=1}^D \sX_i$ is a $D$-dimensional input space, where each individual dimension is either bounded real or categorical, that is, $\sX_i$ is either $[L_i, U_i] \subset \reals$ (with lower and upper bounds $L_i$ and $U_i$, respectively) or $\{1,\,\ldots,\,m_i\}$. 

Associated with $\sX$, there is a DAG structure $\sD$, whose vertices are the dimensions $\{1,\,\ldots,\,D\}$. $\sX$ will be restricted by $\sD$: if vertex $i$ has children under $\sD$, $\sX_i$ must be categorical. $\sD$ is also used to specify whether each input is \emph{active}: that is, relevant to inference about $g$. In particular, we assume each input dimension is only active under some instantiations of its ancestor dimensions in $\sD$. More precisely, we define $D$ functions $\delta_i\colon \sX\mapsto \mathcal{B}$, for $i \in \{1,\,\ldots,\,D\}$, and where $\mathcal{B} = \{\text{true}, \text{false}\}$. We take 
\begin{equation}
 \delta_i(\v{x}) = \delta_i\bigl(\v{x}(\text{anc}_i)\bigr),
\end{equation}
where $\text{anc}_i$ are the ancestor vertices of $i$ in $\sD$, such that $\delta_i(\v{x})$ is true only for appropriate values of those entries of $\v{x}$ corresponding to ancestors of $i$ in $\sD$. We say $i$ is active for $\v{x}$ iff $\delta_i(\v{x})$.

%if all its parent dimensions $P_i$ are active themselves and each parent $p\in P_i$ takes one of the values in the finite set $V_{i,p}$. 
Our aim is to specify a kernel for $g$, \emph{i.e.}, a positive semi-definite function  $k\colon \sX \times \sX \mapsto \reals$. We will first specify an individual kernel for each input dimension, \emph{i.e.}, a positive semi-definite function $k_i\colon \sX \times \sX \mapsto \reals$. $k$ can then be taken as either a sum,
\begin{equation}
 k(\v{x}, \v{x}') = \sum_{i=1}^D k_i(\v{x},\v{x}'),
\end{equation}
product,
\begin{equation}
 k(\v{x}, \v{x}') = \prod_{i=1}^D k_i(\v{x},\v{x}'),
\end{equation}
or any other permitted combination, of these individual kernels. Note that each individual kernel will depend on an  input vector $\v{x}$ only through dependence on both $x_i$ and $\delta_i(\v{x})$,
\begin{equation}
  k_i(\v{x},\v{x}') = \tilde{k}_i\bigl(x_i,\delta_i(\v{x}),x_i', \delta_i(\v{x}') \bigr).
\end{equation}
That is, $x_j$ for $j\neq i$ will influence $k_i(\v{x},\v{x}')$ only if $j \in \text{anc}_i$, and only by affecting whether $i$ is active.

Below we will construct pseudometrics $d{_i}\colon \sX \times \sX \mapsto \reals^+$: that is, $d_i$ satisfies the requirements of a metric aside from the identity of indiscernibles. As for $k_i$, these pseudometrics will depend on an input vector $\v{x}$ only through dependence on both $x_i$ and $\delta_i(\v{x})$. $d{_i}(\v{x}, \v{x}')$ will be designed to provide an intuitive measure of how different $g(\v{x})$ is from $g(\v{x}')$. For each $i$, we will then construct a (pseudo-)isometry $f_i$ from
$\sX$ 
to a Euclidean space ($\reals^2$ for bounded real parameters, and $\reals^m$ for categorical-valued parameters with $m$ choices). That is, denoting the Euclidean metric on the appropriate space as $d{_E}$, $f_i$ will be such that
\begin{equation}
 d{_i}(\v{x},\v{x}')
=
d_{\text{E}}(f{_i}\bigl(\v{x}), f{_i}(\v{x}')\bigr)
\end{equation}
for all $\v{x}, \v{x}' \in \sX$. We can then use our transformed inputs, $f_i(\v{x})$, within any standard Euclidean kernel $\kappa$. We'll make this explicit in the following Proposition. 

Define $\kappa\colon \reals^+ \to \reals$ as a semi-positive definite function over Euclidean space. By the latter, we mean that $K \in \reals^{N\times N}$, defined by 
\begin{equation}
 K_{m, n} = \kappa\bigl(d_{\text{E}}(\v{y}_m, \v{y}_n)\bigr),\quad \text{for }
\v{y}_m, \v{y}_n \in \reals^P,\quad m, n = 1, \ldots, N, 
\end{equation}
is positive semi-definite. A popular example of such a $\kappa$ is the exponentiated quadratic, for which $\kappa(\delta) = \sigma^2 \exp(-\frac{1}{2} \frac{\delta^2}{\lambda^2})$; another popular choice is the rational quadratic, for which $\kappa(\delta) = \sigma^2 (1+\frac{1}{2\alpha} \frac{\delta^2}{\lambda^2})^{-\alpha}$.


\begin{prop}
  $\kappa\bigl(d{_i}(\cdot,\cdot)\bigr)$ is a positive semi-definite covariance function over input space $\sX$. 
\label{prop:cont_psd_proof}
\begin{proof}
We require that $K \in \reals^{N\times N}$ defined by
\begin{align*}
 K_{m, n} & = \kappa\bigl(d{_i}(\v{x}_m,\v{x}_n)\bigr)
,\quad \text{for }
\v{x}_m, \v{x}_n \in \sX,\quad m, n = 1, \ldots, N, 
\\
\intertext{is positive semi-definite. Now, from above,}
K_{m, n} & = \kappa\Bigl(d_{\text{E}}(f{_i}\bigl(\v{x}_m), f{_i}(\v{x}_n)\bigr)\Bigr) \\
& = \kappa\bigl(d_{\text{E}}(\v{y}_m, \v{y}_n)\bigr)
\end{align*}
where $\v{y}_m = f{_i}\bigl(\v{x}_m)$ and similar for $\v{x}_m$. Then, by assumption that $\kappa$ is a positive semi-definite function over Euclidean space, $K$ is positive semi-definite. 
\end{proof}
\end{prop}


%%%%%%%%%%%%%%%%%%%%%%%%%%%%%%%%%%%%%%%%%%%%%%%%%%%%%%%%%%%%%%%%%%%%%%%%%%%%%%%%%%%%%%%%%%%%%%%%%%%%%%%%%%%%%%%%%%%%%%%%%%%%%
\section{Bounded Real Dimensions}
%%%%%%%%%%%%%%%%%%%%%%%%%%%%%%%%%%%%%%%%%%%%%%%%%%%%%%%%%%%%%%%%%%%%%%%%%%%%%%%%%%%%%%%%%%%%%%%%%%%%%%%%%%%%%%%%%%%%%%%%%%%%%

Let's first define $f\br_i$ and $d_i$ for the case that the input $\sX_i=[L_i, U_i]$ is bounded real. We first define a function $d_i\br$ on $\sX$,

\begin{eqnarray}
\nonumber{}d_i\br(\v{x}, \v{x}') & = & \left\{\begin{array}{ll}
\nonumber{} 0 & \textrm{ if } \delta_i(\v{x}) = \delta_i(\v{x}') = \textrm{false}\\
\nonumber{} w_i & \textrm{ if } \delta_i(\v{x}) \neq \delta_i(\v{x}')\\
\nonumber{} w_i \sqrt{2} \sqrt{1 - \cos(\pi \frac{x_i-x_i'}{U_i-L_i})} & \textrm{ otherwise. }\end{array}\right\}.
\end{eqnarray}

\begin{prop}
  $d_i$ is a pseudometric on $\sX$. 
\begin{proof}
The non-negativity and symmetry of $d_i$ are trivially proven. To prove the triangle inequality, consider $\v{x}, \v{x}', \v{x}'' \in \sX$. 

~\\\noindent{}{Case 1}: $\delta_i(\v{x}) = \delta_i(\v{x}') = \textrm{false}$, such that $d_i(\v{x},\v{x}') = 0$. Here, from non-negativity, clearly $d_i(\v{x},\v{x}') = 0 \leq d_i(\v{x},\v{x}'') + d_i(\v{x}',\v{x}'')$.

~\\\noindent{}{Case 2}: $\delta_i(\v{x}) \neq \delta_i(\v{x}')$, such that such that  $d_i(\v{x},\v{x}') = w_i$.  Without loss of generality, assume $\delta_i(\v{x}) = \text{true}$, $\delta_i(\v{x}') = \text{false}$ and $\delta_i(\v{x}'') = \text{true}$. 
\begin{align}
d_i(\v{x},\v{x}'') + d_i(\v{x}',\v{x}'') = d_i(\v{x},\v{x}'')  + w_i
\end{align}
Hence $d_i(\v{x},\v{x}'') + d_i(\v{x}',\v{x}'') \geq w_i = d_i(\v{x},\v{x}')$ by non-negativity.

~\\\noindent{}{Case 3}: $\delta_i(\v{x}) = \delta_i(\v{x}')=\textrm{true}$, such that  $d_i(\v{x},\v{x}') = w_i \sqrt{2} \sqrt{1 - \cos(\pi \frac{x_i-x_i'}{U_i-L_i})}$.  If  $\delta_i(\v{x}'') = \text{false}$,
\begin{align}
d_i(\v{x},\v{x}'') + d_i(\v{x}',\v{x}'') = 2 w_i \geq w_i \sqrt{2} \sqrt{1 - \cos(\pi \frac{x_i-x_i'}{U_i-L_i})} = d_i(\v{x},\v{x}').
\end{align} 
If  $\delta_i(\v{x}'') = \text{true}$, consider the worst possible case in which, without loss of generality, $x_i=L_i$ and $x'_i=U_i$, such that $d_i(\v{x},\v{x}')=2 w_i^2$.  We define the abbreviation $\beta'' = \frac{x''_i-L_i}{U_i-L_i}$, giving
\begin{align}
\bigl(d_i(\v{x},\v{x}'') + d_i(\v{x}',\v{x}'')\bigr)^2
& = 2w_i^2 \Bigl(\sqrt{1 - \cos (\pi \beta'')} + \sqrt{1 - \cos \bigl(\pi (1-\beta'')\bigr)}\Bigr)^2\nonumber\\
&=4 w_i^2 \bigl(1 + \left|\sin \pi \beta'' \right|\bigr)\nonumber\\
&\geq 4 w_i^2 = d_i(\v{x},\v{x}')^2.
\end{align}
Hence, from non-negativity, we have $d_i(\v{x},\v{x}'') + d_i(\v{x}',\v{x}'')\geq d_i(\v{x},\v{x}')$.
\end{proof}
\end{prop}

Now we define an isometric embedding $f_i\br$ of $(\sX, d_i\br)$ into $(\reals^{2},d_{\text{E}})$:

\begin{eqnarray}
\nonumber{}f_i\br(\v{x}) & = & \left\{\begin{array}{ll}
[0,0]^\transpose & \textrm{ if } \delta_i(\v{x}) = \textrm{ false }\\
\nonumber{} w_i [\sin{\pi\frac{x_i}{U_i-L_i}}, \cos{\pi\frac{x_i}{U_i-L_i}}]^\transpose & \textrm{ otherwise.}\end{array}\right\}
\end{eqnarray}

\begin{prop}Embedding $f_i\br$ is an isometric embedding of $(\sX, d_i\br)$ into $(\reals^2, d_{\text{E}})$.
\label{prop:f_d_cont_isometric}
\begin{proof}
Consider two inputs $\v{x},\v{x}' \in \sX$. We need to show that $d_i\br(\v{x},\v{x}') = d_{\text{E}}(f_i\br(\v{x}),f_i\br(\v{x}'))$.
We use the abbreviation $\alpha = \pi\frac{x_i}{U_i-L_i}$ and $\alpha' = \pi\frac{x'_i}{U_i-L_i}$ and consider the following three possible cases of dimension $i$ being active or inactive in $\v{x}$ and $\v{x}'$.

~\\\noindent{}{Case 1}: $\delta_i(\v{x}) = \delta_i(\v{x}') = \textrm{false}$.
In this case, we trivially have 
\[d_{\text{E}}(f_i\br(\v{x}),f_i\br(\v{x}')) = d_{\text{E}}([0,0]^\transpose, [0,0]^\transpose) = 0 = d_i\br(\v{x},\v{x}').\]

~\\\noindent{}{Case 2}: $\delta_i(\v{x}) \neq \delta_i(\v{x}')$. In this case, we have
\[d_{\text{E}}(f_i\br(\v{x}),f_i\br(\v{x}')) = d_{\text{E}}([\sin{\alpha}, \cos{\alpha}]^\transpose, [0,0]^\transpose) = \sqrt{w_i^2 (\sin^2{\alpha} + \cos^2{\alpha})} = w_i = d_i(\v{x},\v{x}'),\]
and symmetrically for $d_{\text{E}}([0,0]^\transpose, [\sin{\alpha}, \cos{\alpha}]^\transpose)$.

~\\\noindent{}{Case 3}: $\delta_i(\v{x}) = \delta_i(\v{x}') = \textrm{true}$. We have:
\begin{eqnarray}
\nonumber{}d_{\text{E}}(f_i\br(\v{x}),f_i\br(\v{x}')) & = & d_{\text{E}}(w_i [\sin{\alpha}, \cos{\alpha}]^\transpose, w_i [\sin{\alpha'}, \cos{\alpha'}]^\transpose)\\ 
\nonumber{}& = & w_i \sqrt{(\sin{\alpha}-\sin{\alpha'})^2+ (\cos{\alpha}-\cos{\alpha'})^2}\\
\nonumber{}& = & w_i \sqrt{\sin^2{\alpha} -2 \sin{\alpha}\sin{\alpha'} + \sin^2{\alpha'}  + \cos^2{\alpha} -2 \cos{\alpha}\cos{\alpha'} + \cos^2{\alpha'} }\\
\nonumber{}& = & w_i \sqrt{(\sin^2{\alpha}+\cos^2{\alpha})   +  (\sin^2{\alpha'}+\cos^2{\alpha'})   -2 (\sin{\alpha}\sin{\alpha'} + \cos{\alpha}\cos{\alpha'})}\\
\label{eqn:simplified}& = & w_i \sqrt{ 1+1-2 \cos(\alpha-\alpha')}\\
\nonumber{}& = & w_i \sqrt{2} \sqrt{1 - \cos(\pi \frac{x_i-x_i'}{U_i-L_i})} = d_i(\v{x}, \v{x}'),
\end{eqnarray}
where (\ref{eqn:simplified}) follows from the previous line by using the identity 
\[\cos{(a-b)} = \cos{a}\cos{b} + \sin{a}\sin{b}.\]
\end{proof}
\end{prop}






%%%%%%%%%%%%%%%%%%%%%%%%%%%%%%%%%%%%%%%%%%%%%%%%%%%%%%%%%%%%%%%%%%%%%%%%%%%%%%%%%%%%%%%%%%%%%%%%%%%%%%%%%%%%%%%%%%%%%%%%%%%%%
\section{Categorical Dimensions}
%%%%%%%%%%%%%%%%%%%%%%%%%%%%%%%%%%%%%%%%%%%%%%%%%%%%%%%%%%%%%%%%%%%%%%%%%%%%%%%%%%%%%%%%%%%%%%%%%%%%%%%%%%%%%%%%%%%%%%%%%%%%%

Next, let's consider a categorical-valued input dimension $i$ with $m_i$ possible values $V_i = \{v_{i,1}, \dots, v_{i,m_i}\}$. As in the continuous case, we define a delta function $\delta_i$ that maps complete inputs $\v{x} \in \sX$ to true if dimension $i$ is active in the context of the instantiation of $i$'s ancestors in $\v{x}$, and to false otherwise. Let $d{_E}^{m_i}$ denote the Euclidean distance metric in $\reals^{m_i}$. We define a metric $d{_i}\cat$ on $\sX$ and an isometric embedding of $(\sX, d{_i}\cat)$ into $(\reals^{m_i},d{_E}^{m_i})$:
\begin{eqnarray}
\nonumber{}d{_i}\cat(\v{x}, \v{x}') & = & \left\{\begin{array}{ll}
\nonumber{} 0 & \textrm{ if } \delta_i(\v{x}) = \delta_i(\v{x}') = \textrm{false}\\
\nonumber{} w_i & \textrm{ if } \delta_i(\v{x}) \neq \delta_i(\v{x}')\\
\nonumber{} w_i \sqrt{2} \indicator_{x_i \neq x_i'} & \textrm{ otherwise. }\end{array}\right\}
\end{eqnarray}

\begin{eqnarray}
\nonumber{}f{_i}\cat(\v{x}) & = & \left\{\begin{array}{ll}
\v{0} \in \reals^{m_i} & \textrm{ if } \delta_i(\v{x}) = \textrm{ false }\\
\nonumber{} w_i \v{e_j} & \delta_i(\v{x}) = \textrm{ true and } x_i = v_{i,j},
\end{array}\right\}
\end{eqnarray}
\noindent{}where $\v{e_j} \in \reals^{m_i}$ is zero in all dimensions except $j$, where it it $1$.

\begin{prop}Embedding $f{_i}\cat$ is an isometric embedding of $(\sX, d{_i}\cat)$ into $(\reals^{m_i},d{_E}^{m_i})$.
\label{prop:f_d_cat_isometric}
\begin{proof}
Consider two inputs $\v{x},\v{x}' \in \sX$. As in the proof of Proposition \ref{prop:f_d_cont_isometric}, we need to show that $d{_i}\cat(\v{x},\v{x}') = d{_E}^{m_i}(f{_i}\cat(\v{x}),f{_i}\cat(\v{x}'))$ and consider the following cases.

~\\\noindent{}{Case 1}: $\delta_i(\v{x}) = \delta_i(\v{x}') = \textrm{false}$.
In this case, we trivially have 
\[d{_E}^{m_i}(f_i\br(\v{x}),f_i\br(\v{x}')) = d{_E}^{m_i}(\v{0}, \v{0}) = 0 = d_i\br(\v{x},\v{x}').\]

~\\\noindent{}{Case 2}: $\delta_i(\v{x}) \neq \delta_i(\v{x}')$. In this case, we have
\[d{_E}^{m_i}(f{_i}\cat(\v{x}),f{_i}\cat(\v{x}')) = d{_E}^{m_i}(w_i \v{e_j}, \v{0}) = w_i = d_i(\v{x},\v{x}'),\]
and symmetrically for $d_{\text{E}}(\v{0}, w_i \v{e_j})$.

~\\\noindent{}{Case 3}: $\delta_i(\v{x}) = \delta_i(\v{x}') = \textrm{true}$. 
If $x_i=x_i'=v_{i,j}$, we have 
\begin{eqnarray}
\nonumber{}d{_E}^{m_i}(f{_i}\cat(\v{x}),f{_i}\cat(\v{x}')) & = & d{_E}^{m_i}(w_i \v{e_j}, w_i \v{e_j}) = 0 = d{_i}\cat(\v{x}, \v{x}')
\end{eqnarray}

\noindent{}If $x_i=v_{i,j} \neq v_{i,j'} = x_i'=$, we have 
\begin{eqnarray} 
\nonumber{}d_{\text{E}}(f{_i}\cat(\v{x}),f{_i}\cat(\v{x}')) & = & d{_E}^{m_i}(w_i \v{e_j}, w_i \v{e_{j'}}) = w_i \sqrt{2} = d{_i}\cat(\v{x}, \v{x}')
\end{eqnarray}
\end{proof}
\end{prop}

\begin{prop}
  $\kappa\bigl(d{_i}\cat(\cdot,\cdot)\bigr)$ is a positive semi-definite covariance function over input space $\sX$. 
\begin{proof}
 This is a trivial extension to the proof of Proposition \ref{prop:cont_psd_proof}.
\end{proof}
\end{prop}

\note{FH: missing - the trivial part where we multiply together all the kernels to define a PSD kernel for the whole space.}

%\bibliographystyle{theapa}
%\renewcommand{\baselinestretch}{0.97}
%\footnotesize{\bibliography{abbrev,frankbib}}


\end{document}

